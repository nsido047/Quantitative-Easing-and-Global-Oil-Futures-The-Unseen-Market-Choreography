
\title{Quantitative Easing Commodities Channels}
\section{Foreign Exchange}
The foreign exchange channel affects oil prices through the devaluation or appreciation of a countries currency from an expansion in long term rates through quantitative easing. The expectation is that if a central bank announces the intention to increase the scale of large scale asset purchases all else equal the country's currency would depreciate against foreign partners. Through the currencies depreciation 

Frankel (2008) discusses the implications of currency fluctuations as it related to oil prices, the author dictates that monetary policy tightening appreciates a currency against the dollar preventing the domestic price of oil from rising. Due to global oil futures being priced in U.S. dollars any monetary policy conducted by the Federal Reserve will have an impact on the spot value of the U.S. dollar thus affecting oil prices (Rosa, 2014). Rosa (2014) finds that upon observing oil movements on days where there is an LSAP Surprise the exchange rate channel cannot explain all the variation in the price of the front-month WTI futures. The author tests the hypothesis of the exchange rate channel by observing the intraday percentage change in the dollar price of crude oil from 10-min before to 50-min after the FOMC LSAP announcement, and the intraday percentage change in the U.S. dollar exchange rate against five currencies (the euro, the British pound, the Canadian dollar, the Swiss franc, and the Japanese yen). 

[Insert Chart of USD/EUR Exchange Rate and Brent Crude Prices (overlay with LSAP dates of ECB)]

A mechanical exchange rate channel would imply that the exchange rate perhaps USD/EUR would exactly offset the change in price of oil specifically WTI with respect to a FOMC LSAP announcement. Miranda-Pinto et al. (2023) hypothesize that all else equal an appreciation of the USD dollar increases the price of commodities in foreign local currency, which then decreases demand and stimulates supply, thus, putting downward pressure on prices. The authors observe that most commodity prices overshoot with respect to the change in the exchange rate in response to U.S. monetary tightening. Soriano and Torro (2022) more recently studies the effect of ECB monetary policy announcement and their effect on high-frequency Brent Crude futures prices dating back to the great recession. The authors results show that the pass-through of monetary policy to the oil Brent price is mostly due to the exchange rate response on event days. Brent Crude futures are priced in USD therefore the natural channel one could hypothesize as having a significant pass through to foreign oil prices would be the change in the USD/EUR. 

Glick and Leduc (2012) observe various great recession period LSAP announcements and their effect on foreign exchange performance. The results for the Bank of England LSAP announcements and their daily effect on the Dollar/GBP movement is quite insignificant for a surprise quantitative easing announcement. The authors find the contrary movement for U.S. Federal Reserve announcements as all observed currency pairs were statistically significant for both LSAP positive and negative surprises. 

\section{Growth Expectation}
The growth expectation channel affects oil futures through the expectation of future economic growth. The logic of the channel is that if the market expects future economic growth to be positive this would benefit oil futures. Yang et al. (2022) observe the expectation channel through GDP expectation as in Jarociński and Karadi (2020) and its effects on oil prices. The unexpected increase in the interest rate can make some market participants more optimistic about future economic growth, empirical results prove that GDP expectation is one of the channels through which central bank information and monetary policy shocks can affect oil prices. Contrary to how market participants may expect that a positive LSAP announcement would depress long-term rates and feed through to commodities, the economic expectation channel passes through in the opposite way. If the central bank announces a quantitative easing taper this would send positive signals about future economic growth thus potentially lowering inflation expectations and boosting oil futures through the expectation of economic expansion.

[Insert table of the change in break-even inflation rate in U.S. and U.K. for initial LSAP announcements (1,2,3,4)]

\section{Finance/Investor Displacement}
This section deals with observing investor displacement that occurs in the asset markets where central banks will direct their action towards such as agency debt, mortgage backed securities or other securitized assets as part of their LSAP programs. Glick and Leduc (2012) discuss this channel they call 'portfolio balance effects' whereby the expectation of central bank purchases reduces the overall supply of longer-term securities available to market participants. The authors discern that "If some investors, such as pension funds or insurance companies, have a preference to hold longer-term securities, these “habitat” preferences make the yields on securities of different maturities partly depend on their relative supplies. As a result, central bank purchases that reduce the stock of long term securities held by the private sector push up the price of these securities, lessen the term premium required to compensate investors to hold them, and hence lower long-term interest rates" (Glick and Leduc, 2012, p. 2080).  Rosa (2014) also examines the finance channel in witch the author refers to as 'portfolio balance' whereby the LSAP programs conducted by the Federal Reserve displaces private investors from the treasury and mortgage backed security market and moves money to other assets classes one of those being commodities.

[Box-plot of comm. and non-comm. open interest change for WTI and Brent for both QE and QT announcements]

\section{Inventory
}The inventory channel interacts with quantitative easing and tightening by the change in the borrowing rate. When both short and long-term rates price in a central bank LSAP announcement the opportunity cost  of holding physical inventory in the oil market. For example when a Central Bank announces the tapering of an LSAP program long-term rates should increase the cost of carrying the physical commodity leading to lower inventories higher supply and in turn lower Oil Prices. 

Quantitative easing typically involves lowering interest rates and injecting liquidity into financial markets. This can reduce the cost of carrying physical commodities like oil, as financing and storage costs decrease. As a result, the commodity convenience yield may decrease because holding physical inventories becomes relatively less valuable compared to financial assets. The perception of scarcity or immediate demand for physical commodities may diminish under quantitative easing conditions. In contrast under quantitative tightening announcements made by the central bank the convenience yield would increase as a result of the higher carrying cost of the commodity. The increase in the convenience yield could be due to higher financing costs can reduce the incentive to hold physical oil inventories, thereby increasing the premium (convenience yield) associated with holding physical oil. 


