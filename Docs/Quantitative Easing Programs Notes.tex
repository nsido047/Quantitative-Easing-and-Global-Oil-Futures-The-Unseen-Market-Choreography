\title{Quantitative Easing Program Breakdown}
\section{Federal Reserve}

The Federal Reserve implemented LSAP1 in late 2008 and concluded the first round of QE at the end of 2009. In November of 2008 The Federal Open Market  (FOMC) announced the intention to purchase 500 billion USD of agency MBSs and 100 billion USD of agency debt with the prospect of continuing into the fall of 2009 (Krishamurthy et al., 2011). In March of 2009 the FOMC made the formal announcement of the further purchase of up to an additional 750 billion of agency mortgage-backed securities USD, bringing its total purchases of these securities to up to 1.25 trillion USD in the year, and to increase its purchases of agency debt this year by up to 100 billion USD to a total of up to 200 billion USD.  Moreover, to help improve conditions in private credit markets, the Committee decided to purchase up to 300 billion USD of longer-term Treasury securities further into the year. LSAP1 was eventually rolled into LSAP2 in November of 2010 2-years after the formal announcement of the first QE program. Stroebel and Taylor (2012) find that during the LSAP1 implementation in 2008 the Federal Reserve had made a considerable expansion to its balance sheet by engaging in foreign currency swaps, providing money market funds and banks with excess reserves. During LSAP1 these assets were zeroed out with the purchase of mortgage-backed securities.

\textit{[Insert Chart with Federal Reserve Balance Sheet Expansion from 2008-2010]}

The introduction of LSAP2 in November 2010 focused on expanding the purchase program to include treasuries. The FOMC made the formal announcement on November 3rd with the intention to purchase a further 600 billion USD of longer-term Treasury securities by the end of the second quarter of 2011, at a pace of 75 billion USD per month. Less than a year later in September of 2011 the FOMC made the reintroduction of Operation Twist for the first time since 1961. The policy involved selling 400 billion in short-term Treasuries in exchange for the same amount of longer-term bonds, starting in October 2011 and ending in June 2012. The process was proposed that "by purchasing 400 billion USD worth of Treasury bonds with maturities of 6–30 years and selling bonds with maturities of less than 3 years, the FOMC intended to extend the average maturity of the Fed’s portfolio" (Krishamurthy et al., 2011, p. 337).

Not long after LSAP2 was introduced in September of 2012. LSAP3 targeted MBS once again similar to LSAP2 and was focused on "monthly purchase of \$85 billion through the purchase of mortgage-backed securities (\$40 billion) and longer-term Treasury securities (\$45 billion)" (Krishamurthy et al., 2011, p. 337). The FOMC cited the slow growth in employment and elevated unemployment rate in late 2012 as the driving force behind the decision to implement LSAP3. The central bank later announced the tapering of QE3 in December 2013 whereby it would lower its monthly long-term Treasury bond purchases to \$40 billion and mortgage-backed securities to \$35 billion a month, both reductions of \$5 billion. In September of 2014 the FOMC published its Policy Normalization Principles and Plans, in which the Committee layed out its plans to reduce its holdings in a gradual and predictable manner primarily by ceasing to reinvest repayments of principal on securities held in the SOMA. 

By far the largest nominal program the Federal Reserve introduced was in March 2020 in response to the COVID-19 induced economic lock downs. On March 15th the FOMC announced the intent to purchase  500 billion USD of long-term treasuries and 200 billion USD of MBS totalling 750 billion of expected easing. The desk was instructed to complete these purchases with no explicit timeline but rather at a pace appropriate to ensure smooth continuity in the treasury and MBS markets. According to Levin and Co-Authors LSAP4 was originally aimed at mitigating the lack of liquidity in the MBS and agency debt markets it shifted into a broader monetary stimulus program. "From mid-March 2020 to the end of March 2022, the FOMC purchased about \$4.6 trillion in Treasuries and agency MBS, funding those purchases through a corresponding increase in bank reserves and overnight reverse repos" (Levin et al., 2022, p. 1). The authors also find that LSAP4 details were opaque as one can see as soon after the first introduction of LSAP4. On March 23rd 2024 one week after the initial introduction of LSAP4 the FOMC instructed the System Open Market Account (SOMA) desk to conduct operations at an unlimited quantity including introducing the purchase of commercial agency MBS to ensure smooth continuity of markets with no explicit pace or timeline mentioned. The pace in the initial months of the program eclipsed LSAP3 as "over the four weeks from 18 March to 15 April 2020, the SOMA expanded its holdings of agency MBS by about \$225 billion and its holdings of Treasury notes and bonds by about \$1.3 trillion. In effect, the Fed’s securities purchases within that four-week period were nearly as large as the total amount of purchases made during QE3 in 2012-14" (Levin et al., 2022, p. 7).

While amounts of timelines differed from the latter QE programs to the great financial crisis easing the pace of tapering by the FOMC differed greatly from LSAP4 and LSAP3. LSAP4 tapering began in early November of 2021 at pace of 10 billion USD in Treasuries and 5 billion USD in MBS each month. The tapering in 2021 was double the pace of that seen in LSAP3 considering also the nominal amount was close to a trillion USD compared to the relatively inconsequential amount of LSAP3.

\textit{[Insert Chart of Tapering Pace of LSAP4 vs. LSAP3]}

\textit{[Insert Table of Breakdown of LSAP1 - LSAP4]}

---
\subsection{QE1 (2008-2010)}
From: Krishnamurthy et al. (2012)
\begin{quote}
"Gagnon and others (2010) provide an event study of QE1 based on the announcements of long-term asset purchases by the Federal Reserve in the period from late 2008 to 2009. QE1 included purchases of MBSs, Treasury securities, and agency securities. Gagnon and others (2010) identify eight event dates beginning with the November 25, 2008, announcement of the Federal Reserve’s intent to purchase 500 billion USD of agency MBSs and 100 billion USD of agency debt and continuing into the fall of 2009. 
\end{quote}
From: Belke et al. (2017)
\begin{quote}
"In November 2008, the Federal Reserve announced the first round of asset purchases (QE1). These purchases were to include government-sponsored enterprise (GSE) debt and agency mortgage-backed securities (MBSs) of up to 600 billion USD. The motivation given was that the spread on agency bonds had increased, thus making house purchases more expensive. After announcing the intention to extend the program in January 2009, the Federal Open Market Committee (FOMC) decided to purchase an additional 750 billion USD in (agency) MBSs, 100 billion USD in agency debt, and also started to purchase long-term Treasury securities worth 300 billion USD in March 2009. These three different announcements explain why it is not possible to assign a unique date to the start of QE1. We keep to the widely accepted practice of dating the announcement of QE1 to November 2008. In total, the Fed purchased assets worth 1.75 trillion USD between November 2008 and March 2010. Its balance sheet more than doubled over this period (see also Borio and Zabai, 2016)."
\end{quote}

\subsection{QE2 (2010-2011)}
From: Krishnamurthy et al. (2012)
\begin{quote}
"In October 2010, the FOMC announced the second round of QE (QE2). It contained purchases of 600 billion USD worth of treasuries and was finished in June 2011. A few months later, the implementation of a maturity extension program, the so-called Operation Twist 2 (OT, was launched (Borio and Zabai, 2016). By purchasing 400 billion USD worth of Treasury bonds with maturities of 6–30 years and selling bonds with maturities of less than 3 years, the FOMC intended to extend the average maturity of the Fed’s portfolio."
\end{quote}
\subsection{QE3 (2012-2014)}
From: Krishnamurthy et al. (2012)
\begin{quote}
"Eventually, the third round of QE (QE3) started in September 2012. It targeted a monthly purchase of \$85 billion through the purchase of mortgage-backed securities (\$40 billion) and longer-term Treasury securities (\$45 billion). In contrast to the other programs, the continuation of QE3 was tied to the improvement in the labour market. Overall, the Fed balance sheet increased by about \$3.5 trillion (roughly 20\% of GDP). As shown in Fig. 1, the balance sheet of the Federal Reserve has now reached over \$4 trillion, or close to 25\% of GDP. Two assets dominate the asset side: Treasury securities (about \$2.5 trillion) and federal agency securities (\$22.4 billion).7 The latter are all guaranteed by the federal government of the United States. It is thus formally true that the Federal Reserve has intervened in the market for securitized mortgages, but it has bought only securities guaranteed by the government. In terms of the evolution of the balance sheet, one can clearly see the impact of QE 1, 2 and 3".
\end{quote}
\subsection{QE4 (2020-2022)}
From: Levin et al. (2022)

\begin{quote}
"...leading to a doubling of the Fed’s securities holdings to about \$8.5 trillion as of March 2022. QE4 was initially aimed at mitigating strains in markets for Treasuries and agency mortgage-backed securities but was subsequently aimed more broadly at supporting market functioning and providing monetary stimulus"
\end{quote}
\begin{quote}
"Over the four weeks from 18 March to 15 April 2020, the SOMA expanded its holdings of agency MBS by about \$225 billion and its holdings of Treasury notes and bonds by about \$1.3 trillion. In effect, the Fed’s securities purchases within that four-week period were nearly as large as the total amount of purchases made during QE3 in 2012-14".
\end{quote}
\begin{quote}
"By summer 2021, however, it became evident that the tapering of QE4 would need to begin soon and at a more rapid pace than for QE3. At its July meeting, the FOMC finally acknowledged that “the economy has made progress” but refrained from characterizing such progress as “substantial.”22 In September, policymakers stated that “a moderation in the pace of asset purchases may soon be warranted.” In early November, the FOMC began tapering its purchases and signaled that the process would likely be completed over a six month period – about twice the speed of the tapering of QE3.24 By December, even that timeline was judged to be insufficiently rapid; the FOMC accelerated the pace of tapering and indicated that it would be completed by March 2022.".
\end{quote}
\begin{quote}
"In May 2022 the FOMC announced that it would begin shrinking the SOMA account during the following month by setting target amounts for the rolloff of Treasury securities and caps on the rolloff of agency MBS.26 In particular, its holdings of Treasuries would decline by \$120 billion in 2022:Q2 and at a quarterly rate of \$180 billion thereafter; those declines would predominantly reflect maturing Treasury notes and bonds but could be augmented by allowing maturing Treasury bills to roll off instead of being reinvested into new Treasury bills"
\end{quote}
---
\section{Bank of England}

The Bank of England (BoE) followed the Federal Reserve closely in the implementation of Quantitative Easing and was one of the largest programs in proportion to a nation's nominal GDP. The BoE followed the U.S. swiftly in 2009 with the introduction of the first rounf of QE. In March of 2009 the Monetary Policy Committee (MPC) announced that it would purchase £75 billion of assets over three months funded by central bank reserves, with conventional bonds likely to constitute the majority of purchases. Gilt purchases were to be restricted to bonds with a residual maturity of between 5 and 25 years (Glick and Leduc, 2012).  The MPC later announced the program was to be extended by an additional £50 billion to a total of £125 billion, these purchases were intended to still remain in the scope of long-dated gilts but private sector assets were eventually included. Lyonnett and Werner (2012) find that even prior to the official March 2009 QE announcement the Special Liquidity Scheme in April 2008 closely resembled conventional quantitative easing whereby financial institutions could swap illiquid assets for short-dated gilts for up to three years, because these were lending transactions they were not reported on the Bank of England balance sheet. 

The MPC would further expand QE1 two additional times in August 2009 (£50 billion) and again in November of the same year (£25 billion). The Bank cited rising deflation and falling output in 2009 as the driver of the multiple expansions of QE1. The Central Bank would cumulatively purchase an estimated £175 billion in assets by the end of October 2009 and would include gilts with a residual maturity extending beyond three years. The MPC would complete QE1 extending the final round of purchases to £200 billion in November 2009. The pace of QE1 was the fastest until LSAP 5 from the Bank of England at an implied speed of £6bn per week on average, QE2-4 were conducted at roughly half the pace of QE1.

\textit{[Insert Table of Early 2009 MPC Announcements]}

QE2 was introduced on October 6th 2011 roughly two years after the conclusion of QE1. QE2 was introduced with the intent to purchase above market expectations of £75 billion in gilts over a four month time frame. QE2 had no specific mention of maturity of gilt purchases and was extended once more in February 2012 to increase the total program purchases by an additional £50 billion of gilts. The Bank cited slower economic growth at home and abroad, especially in the UK's main export markets, as well as problems in the euro zone, and strains on the banking system. 

QE3 and QE4 were one time announcements in July of 2012 and August 2016 respectively. In July 2012 soon after the last announcement of QE2 the Bank of England's monetary policy committee voted to raise the total amount of quantitative easing to £375 billion an increase of a further £50 billion of asset purchases. Finally in 2016 proceeding the vote from the U.K. to part from the EU, the MPC decreased its growth forecast and launched a £70bn bond-buying programme. The purchase of up to £10 billion of UK corporate bonds; and an expansion of the asset purchase scheme for UK government bonds of £60 billion. QE4 had a longer-time horizon of 6 months compared to the historical average program timeline of 3-4 months. QE4 was the first program introduced by the MPC that would include corporate debt which had not been purchased in the previous three programs. The decisions from the Monetary Policy Committee came after a series of negative economic data in the wake of the vote to leave the European Union in June. The MPC also introduced a new Term Funding Scheme, up to £100 billion funded from newly-printed money to ensure the pass through of the subsequent rate cut onto businesses and consumers.

The Central Bank's most recent QE program would follow the Federal Reserve's actions in the wake of the COVID-19 pandemic. Three distinct QE5 announcements were made in 2020 followed by two intentions to taper the program in early and late 2022. In March 2020 the MPC announced the intent to increase the bank's holdings of bonds by £200 billion, financed by printing money. Later in June the total asset purchases were to be increased by £100 billion to further aid the recovery. Finally in November of that same year the MPC would announce the intent to purchase a final additional round of £150 of gilts amid the second European COVID-19 lockdown.

The MPC would provide few detail at their February meeting in 2022 about the intended taper of QE5. The explicit announcement of the taper would come later in the year in September. The MPC voted unanimously to reduce the stock of purchased gilts, financed by the issuance of central bank reserves, by £80 billion over the next twelve months, to a total of £758 billion, in line with the strategy set out in the minutes of the August MPC meeting.


---
\subsection{QE1 (2009)}
From: Glick and Leduc (2012)
\begin{quote}
"Monetary Policy Committee (MPC) announces that it would purchase £75 billion of assets over three months funded by central bank reserves, with conventional bonds likely to constitute the majority of purchases. Gilt purchases were to be restricted to bonds with a residual maturity of between 5 and 25 years".
\end{quote}
From: Lyonnett and Werner (2012)
\begin{quote}
"The Special Liquidity Scheme was introduced in April 2008, allowing banks and building societies to swap some of their illiquid assets (notably asset-backed securities) for liquid UK Treasury bills for a period of up to three years. As these trades are lending transactions they remain off-balance sheet"
\end{quote}
\subsection{QE2 (2011-2012)}

\subsection{QE3 (2012)}

\subsection{QE4 (2016)}
\begin{quote}
"Introduced a new Term Funding Scheme, up to £100 billion funded from newly-printed money, designed to ensure banks pass on the interest rate cut by giving banks access to cheap loans linked to the amount they lend firms and households"
\end{quote}
\subsection{QE5 (2020-2022)}
\begin{quote}
"Drawing on survey data on market expectations, \href{https://bankunderground.co.uk/2021/03/26/what-to-expect-when-theyre-expecting/}{\textbf{Froemel et al (2021)Opens in a new window}} infer that the response of yields to the 2020 QE announcements was consistent with medium term expectations of the stock of purchases responding to news about the pace of purchases. Therefore, all else equal, a lower than expected announced pace will be associated with a lower expected purchase stock in the medium term, and to potential upward pressure on yields, and vice versa"
\end{quote}
---
\section{European Central Bank}

The European Central Bank (ECB) formally began quantitative easing in early 2015 with the introduction of the Asset Purchase Programme. Prior to this the monetary authority purchased covered bonds under the Securities Market Programme (SMP) which was not explicitly defined as quantitative easing. While the SMP was primarily used to bolster market liquidity during the Eurozone Sovereign Debt Crisis that began in 2010 the program was in fact reflective of early quantitative actions by the ECB.

The SMP was introduced due to the lack of demand for Soverign Debt from Eurozone nations such as Spain, Italy and Greece. The program details were announced by the ECB on March 10, 2010 and was concluded on September 6, 2012 and was replaced by the Outright Monetary Transactions (OMT) program. At that time, purchases made under the SMP totaled €218 billion. The program which was considered a surprise by market participants had two formal announcements to the public, one on March 10, 2010 which detailed the program without explicit mention of the timeline and the second on August 7, 2011 detailing a reactivation an expansion which included Italy and Spain in the sovereign debt purchases.  

While there was formal announcement of the program there was no explicit mention of the key features such as the targeted securities, maturities , amount or timeline intended by the ECB. After the conclusion of the program it was reported that over half of the securities purchased under the SMP were Italian government bonds, roughly €102.8 billion (nominal).

A 2013 study from the ECB explains the difference between direct monetary stimulus, such as quantitative easing, and the 'non-standard' measures introduced by the ECB in 2011. The authors describe these measures as complementary to standard interest rate decisions, aimed at supporting the effective transmission of monetary policy to the economy rather than providing additional direct monetary stimulus. They note that while typical quantitative easing programs, like those in the U.S. and U.K. during the Great Recession, resulted in central bank balance sheets increasing by over 150\%, the ECB's balance sheet grew by only 50\%. Despite the smaller increase, the ECB still conducted open market purchases of €40 billion in eurozone covered bonds after the conclusion of the Securities Markets Programme (SMP) in late 2011. The scale of balance sheet expansion does not change the nature of the intervention. Both involve large-scale asset purchases aimed at increasing liquidity and stimulating the economy. The ECB's focus on lending against collateral rather than purchasing assets outright is cited as a distinguishing factor. However, the purchase of €40 billion in eurozone covered bonds after the SMP's conclusion still represents direct market intervention similar to QE. The effect of such purchases on market liquidity and asset prices aligns with conventional QE objectives. 

In October 2014, the ECB formally discussed QE, agreeing on the details of a QE program for Asset-Backed Securities and Covered Bonds, known as the Asset Purchase Programme. On January 22, 2015, the ECB announced it would purchase securities at a pace of €60 billion per month until at least September 2016. This was the first time the ECB had announced a purchase program with an explicit amount, pace, and timeline. The goal was to continue the program until the Governing Council observed a sustained adjustment in the path of inflation consistent with achieving rates below, but close to, 2\% over the medium term. The ECB would buy bonds issued by euro area central governments, agencies, and European institutions in the secondary market using central bank money, which the selling institutions could then use to buy other assets and extend credit to the real economy. This approach aimed to ease financial conditions. 

The APP concluded in December of 2018 after three years of security purchases. The program resumed in 2020 following the global closures as a result of the COVID-19 pandemic. 

\begin{itemize}
    \item €60 billion of net purchases from March 2015 to March 2016;
    \item €80 billion of net purchases from April 2016 to March 2017;
    \item €60 billion of net purchases from April to December 2017;
    \item €30 billion of net purchases from January to September 2018;
    \item €15 billion of net purchases from October to December 2018
\end{itemize}

 
\subsection{SMP (2010-2011)}
\begin{quote}
"SMP purchases were conducted by Eurosystem central banks and proceeded in two main waves. The first dealt with government bonds from the secondary markets of Greece, Ireland, and Portugal, and the second (which began on August 7, 2011, after a period of inactivity) expanded the program’s jurisdiction to deal with government bonds from Italy and Spain. During the second wave, central banks continued to purchase securities from Ireland and Portugal, but Greece was not included in this new wave of transactions. The program ended on September 6, 2012 and was replaced by the Outright Monetary Transactions (OMT) program. At that time, purchases made under the SMP totaled €218 billion. All bonds purchased under the SMP were to be held until maturity; as of October 26, 2018, the SMP portfolio held nearly €73 billion at amortized cost." (Smith 2020).
\end{quote}
\begin{quote}
"The ECB managed SMP purchases, while the Eurosystem central banks conducted them; central banks’ purchases were allocated according to their share of the ECB’s capital (ECB5). The purchases proceeded in two main waves, referred to here as SMP-1 and SMP-2. The purchases made in SMP-1 dealt with government bonds from the secondary markets of Greece, Ireland, and Portugal. SMP-2 began when the program was relaunched on August 7, 2011, after a brief period of inactivity; this second wave of purchases expanded the program’s jurisdiction to Italy and Spain, while discontinuing purchases from Greece5 (Ghysels et al. 2017). Overall, the program ran for a nearly two years"(Smith, 2020).
\end{quote}
\subsection{OMT (2012-2014)}
\begin{quote}
"The question is sometimes raised whether the ECB’s non-standard measures are tantamount to quantitative easing and equally amount to ‘printing money’. A number of observations show that this was not the case, at least in the first stages of the crisis before its epicentre clearly moved to the euro area and even though in terms of their effects on the economy the two approaches have similarities, as is seen in section V(iii). First, the quantitative response has been different. During the phase of the global financial crisis (2008–11) before the re-intensification of the euro area sovereign debt crisis, the epicentre of the crisis was in the US and the UK. In line with this and with the choice of the policy response, the balance sheets of the US Federal Reserve and Bank of England increased by about 150\%, whereas that of the Eurosystem only rose by about 50\% during the period (see Figure 4). As the epicentre of the crisis then moved to the euro area, the Eurosystem balance sheet has been significantly widening, similar to the cases of the other two central banks. Second, with the exception of the purchase programmes – which to date have been relatively limited in scope (the CBPP and the SMP amounting in 2012 to about 3\% of GDP) –, the ECB has focused on lending against collateral rather than on purchasing assets. With such lending operations the ECB has not sought to boost market prices. Rather, it has taken for the collateral market prices as given (with daily adjustments to price fluctuations) and, moreover, applied a haircut that can be considerable for some asset classes.31 Third, at least until late 2011 the ECB made clear that non-standard measures were not the new regime but exceptional means that were temporarily required and should be measured in dimensions and phased out as soon as possible. Phasing-out is actually embedded as a design feature of the bulk of the ECB’s non-standard measures. One of the reasons for this more measured approach was the need, in the EMU context, to avoid moral hazard on the side of the various governments" (Thimann & Winkler, 2013).
\end{quote}
